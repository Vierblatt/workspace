\documentclass[12pt,a4paper]{article}
\usepackage[UTF8]{ctex}
\usepackage{graphicx}
\usepackage{geometry}
\usepackage{amsmath}
\usepackage{listings}
\usepackage{xcolor}

% 页面设置
\geometry{left=2.5cm,right=2.5cm,top=2.5cm,bottom=2.5cm}

% 代码样式设置
\lstset{
    language=C++,
    keywordstyle=\color{blue},
    commentstyle=\color{gray},
    stringstyle=\color{red},
    basicstyle=\ttfamily\small,
    breaklines=true,
    numbers=left,
    numberstyle=\tiny\color{gray}
}

\begin{document}

% 封面部分
\begin{center}
\includegraphics[width=1.02431in,height=1.01319in,alt={广东工业大学校徽}]{media/image1.png}

\includegraphics[width=3.13333in,height=0.88333in,alt={广东工业大学校名}]{media/image2.png}

\section*{\textbf{\fontsize{26pt}{30pt}\selectfont 数据结构实验报告}}
\label{sec:data_structure_lab_report}
\vspace{0.5cm}
\textbf{\fontsize{22pt}{26pt}\selectfont 题目:基于红黑树的哈希表实现}

\vspace{2cm}
\begin{tabular}{ll}
\fontsize{16pt}{19pt}\selectfont 学  院: & \fontsize{16pt}{19pt}\selectfont 国际教育学院\\
\fontsize{16pt}{19pt}\selectfont 专  业: & \fontsize{16pt}{19pt}\selectfont 计算机科学与技术国际\\
\fontsize{16pt}{19pt}\selectfont 年级班别: & \fontsize{16pt}{19pt}\selectfont 一班\\
\fontsize{16pt}{19pt}\selectfont 学  号: & \fontsize{16pt}{19pt}\selectfont 3124009862\\
\fontsize{16pt}{19pt}\selectfont 学生姓名: & \fontsize{16pt}{19pt}\selectfont 杨恒熠\\
\fontsize{16pt}{19pt}\selectfont 指导教师: & \fontsize{16pt}{19pt}\selectfont 李小妹\\
\fontsize{16pt}{19pt}\selectfont 编  号: & \fontsize{16pt}{19pt}\selectfont \dotfill\\
\fontsize{16pt}{19pt}\selectfont 成  绩: & \fontsize{16pt}{19pt}\selectfont \dotfill\\
\end{tabular}
\vfill
\textbf{2025年11月}
\end{center}

% 评分部分
\newpage
\begin{center}
\textbf{报告:}

\textbf{报告内容:} □详细  □完整  □基本完整 □不完整

\textbf{设计方案:} □非常合理  □合理  □基本合理 □较差

\textbf{算法实现:} □全部实现  □基本实现  □部分实现 □实现较差

\textbf{测试样例:} □完备  □比较完备  □基本完备 □不完备

\textbf{文档格式:} □规范  □比较规范  □基本规范 □不规范

\vspace{1cm}
\textbf{答辩:}

□理解题目透彻,问题回答流利

□理解题目较透彻,回答问题基本正确

□部分理解题目,部分问题回答正确

□未能完全理解题目,答辩情况较差

\vspace{1cm}
\textbf{总评成绩:}

□优   □良   □中   □及格   □不及格
\end{center}

% 正文部分
\newpage
\section{实验目的}
1. 理解哈希表的基本原理及实现机制,掌握哈希函数的设计方法与冲突解决策略。

2. 深入学习红黑树的特性、平衡维护机制及基本操作(插入、删除、查找),理解其在处理动态数据时的性能优势。

3. 实现基于红黑树解决哈希冲突的哈希表结构,验证其功能正确性并分析其时间复杂度。

4. 培养数据结构设计与实现能力,提升对复杂数据结构组合应用的理解。

\section{实验原理}
\subsection{哈希表}
哈希表(Hash Table)是一种通过键(Key)直接访问数据存储位置的数据结构。其核心思想是通过哈希函数将键映射到表中的索引位置,从而实现快速的插入、删除和查找操作。

哈希函数是哈希表的核心组件,本实验采用取模运算作为哈希函数:
\[
\text{hash}(key) = (key \mod \text{tableSize})
\]
其中,\(tableSize\) 为哈希表的容量(桶的数量)。若计算结果为负数,则通过加 \(tableSize\) 确保索引为非负值。

当不同的键通过哈希函数映射到同一索引时,会产生哈希冲突。本实验采用红黑树作为每个桶(Bucket)的底层数据结构来解决冲突,即每个索引位置对应一棵红黑树,所有映射到该索引的键值对均存储在对应的红黑树中。

\subsection{红黑树}
红黑树是一种自平衡二叉查找树,它通过维护以下特性保证树的高度始终为 \(O(\log n)\):
1. 每个节点要么是红色,要么是黑色。
2. 根节点必须是黑色。
3. 所有叶子节点(NIL节点,空节点)均为黑色。
4. 若一个节点是红色,则其两个子节点必须是黑色(不存在连续的红色节点)。
5. 从任意节点到其所有叶子节点的路径中,包含相同数量的黑色节点。

这些特性确保红黑树的插入、删除和查找操作的时间复杂度均为 \(O(\log n)\),其中 \(n\) 为树中节点的数量。

红黑树的平衡维护主要通过以下操作实现:
- 旋转(左旋和右旋):调整节点的位置关系,不改变二叉查找树的性质。
- 颜色调整:通过修改节点颜色,配合旋转操作维持红黑树的特性。

\section{实验环境}
- 操作系统:Windows 10 64位 / Linux Ubuntu 20.04
- 编译环境:GCC 9.4.0
- 开发工具:Visual Studio Code
- 编程语言:C++

\section{实验内容与步骤}
\subsection{数据结构设计}
1. **红黑树节点结构**:
```cpp
enum Color { RED, BLACK };

template \<typename K, typename V\>
struct RBNode {
    K key;          // 键
    V value;        // 值
    Color color;    // 颜色
    RBNode* left;   // 左孩子
    RBNode* right;  // 右孩子
    RBNode* parent; // 父节点

    RBNode(K k, V v) : key(k), value(v), color(RED), 
                      left(nullptr), right(nullptr), parent(nullptr) {}
};
```