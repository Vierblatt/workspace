\documentclass[12pt,a4paper]{article}
\usepackage[UTF8]{ctex}
\usepackage{geometry}
\usepackage{listings}
\usepackage{xcolor}
\usepackage{titlesec}
\usepackage{graphicx}
\usepackage{amsmath}
\usepackage{fancyhdr}
\usepackage{hyperref}

% 页面设置
\geometry{left=2.5cm,right=2.5cm,top=2.5cm,bottom=2.5cm}

% 标题格式设置
\titleformat{\section}{\Large\bfseries}{\thesection}{1em}{}
\titleformat{\subsection}{\large\bfseries}{\thesubsection}{1em}{}

% 页眉页脚设置
\pagestyle{fancy}
\fancyhf{}
\fancyhead[L]{银行业务模拟系统课程设计报告}
\fancyhead[R]{\thepage}
\fancyfoot[C]{数据结构课程设计}
\setlength{\headheight}{15pt}

% 代码样式设置
\definecolor{codeblue}{rgb}{0.25,0.5,0.5}
\definecolor{codegray}{rgb}{0.5,0.5,0.5}
\definecolor{codepurple}{rgb}{0.58,0,0.82}
\definecolor{backcolour}{rgb}{0.95,0.95,0.92}

\lstdefinestyle{mystyle}{
    backgroundcolor=\color{backcolour},
    commentstyle=\color{codegray},
    keywordstyle=\color{codeblue},
    numberstyle=\tiny\color{codegray},
    stringstyle=\color{codepurple},
    basicstyle=\ttfamily\footnotesize,
    breakatwhitespace=false,
    breaklines=true,
    captionpos=b,
    keepspaces=true,
    numbers=left,
    numbersep=5pt,
    showspaces=false,
    showstringspaces=false,
    showtabs=false,
    tabsize=2
}

\lstset{style=mystyle}

\begin{document}

% 封面
\begin{center}
\includegraphics[width=0.15\textwidth]{media/logo.png}

\vspace{0.5cm}

\textbf{\Large 数据结构课程设计报告}

\vspace{1.0cm}

\textbf{\Huge 银行业务模拟系统}

\vspace{1.5cm}

\begin{tabular}{ll}
\textbf{学\quad 院:} & 计算机学院 \\
\textbf{专\quad 业:} & 计算机科学与技术 \\
\textbf{年\quad 级:} & 2022级 \\
\textbf{学\quad 号:} & 3124009862 \\
\textbf{学生姓名:} & 杨恒熠 \\
\textbf{指导教师:} & 李小妹 \\
\textbf{提交日期:} & 2025年11月
\end{tabular}

\vspace{2.0cm}

\end{center}

\newpage

% 目录
\tableofcontents
\newpage

% 正文
\section{问题描述}
根据《数据结构实验指导书》题目4的要求,设计并实现一个银行业务的事件驱动模拟系统,通过模拟方法求出客户在银行内逗留的平均时间。

\subsection{业务规则}
\begin{enumerate}
\item 客户业务分为两种:
  \begin{itemize}
  \item 第一种:申请从银行得到一笔资金(取款或借款)
  \item 第二种:向银行投入一笔资金(存款或还款)
  \end{itemize}
  
\item 银行有两个服务窗口,相应地有两个队列:
  \begin{itemize}
  \item 客户到达银行后先排第一个队
  \item 处理每个客户业务时,如果属于第一种业务且申请额超出银行现存资金总额而得不到满足,则立刻排入第二个队等候,直至满足时才离开银行;否则业务处理完后立刻离开银行
  \end{itemize}
  
\item 每接待完一个第二种业务的客户,则顺序检查和处理第二个队列中的客户,对能满足的申请者予以满足,不能满足者重新排到第二个队列的队尾

\item 营业时间结束时所有客户立即离开银行
\end{enumerate}

\section{系统设计}
\subsection{架构设计}
系统采用事件驱动架构,通过事件队列管理所有客户到达和离开事件,按时间顺序处理。系统主要包含以下几个模块:

\begin{enumerate}
\item \textbf{事件管理模块}:负责事件的创建、排序和调度
\item \textbf{队列管理模块}:管理客户队列,包括正在等待服务的客户队列和因资金不足而等待的客户队列
\item \textbf{客户管理模块}:负责客户的创建、销毁和业务处理
\item \textbf{银行系统模块}:协调各个模块,维护银行状态
\item \textbf{日志系统模块}:记录系统运行过程中的关键信息
\end{enumerate}

\subsection{数据结构设计}
\subsubsection{客户结构体}
\begin{lstlisting}[language=C,caption=客户结构体]
typedef struct Customer {
    int id;                 // 客户ID
    int arrive_time;        // 到达时间
    int duration;           // 服务时间
    int amount;             // 交易金额(正数表示存款,负数表示取款)
    struct Customer* next;  // 链表指针
} Customer;
\end{lstlisting}

\subsubsection{事件结构体}
\begin{lstlisting}[language=C,caption=事件结构体]
typedef struct Event {
    int occur_time;              // 事件发生时间
    EventType event_type;        // 事件类型
    struct Customer* customer;   // 相关客户
    struct Event* next;          // 链表指针
} Event;
\end{lstlisting}

\subsubsection{队列结构体}
\begin{lstlisting}[language=C,caption=队列结构体]
typedef struct Queue {
    Customer* front;  // 队首指针
    Customer* rear;   // 队尾指针
    int size;         // 队列大小
} Queue;
\end{lstlisting}

\subsubsection{银行系统结构体}
\begin{lstlisting}[language=C,caption=银行系统结构体]
typedef struct BankSystem {
    int total_money;        // 银行总资金
    int current_time;       // 当前时间
    int close_time;         // 关门时间
    int customer_count;     // 客户计数器
    Queue queue1;           // 第一个队列(正在等待服务的客户)
    Queue queue2;           // 第二个队列(因资金不足而等待的客户)
    Event* event_list;      // 事件列表
    int total_customers;    // 总客户数
    long long total_wait_time;  // 总等待时间
    FILE* log_fp;           // 日志文件指针
    LogLevel log_level;     // 日志级别
} BankSystem;
\end{lstlisting}

\subsection{核心算法}
\subsubsection{事件驱动模拟}
系统采用事件驱动的方式进行模拟,主要处理两类事件:
\begin{enumerate}
\item 客户到达事件
\item 客户离开事件
\end{enumerate}

事件按照发生时间顺序存储在事件列表中,每次取出最早发生的事件进行处理。

\subsubsection{客户服务流程}
\begin{enumerate}
\item 客户到达时,加入第一个队列
\item 队列首客户接受服务,创建离开事件
\item 根据业务类型和银行资金情况决定客户去向:
  \begin{itemize}
  \item 存款业务:直接完成,更新银行资金
  \item 取款业务:资金充足则完成,资金不足则加入等待队列
  \end{itemize}
\item 存款业务完成后,检查并处理等待队列中的客户
\end{enumerate}

\subsubsection{等待队列处理}
每当有存款业务完成时,系统会检查等待队列中的客户:
\begin{enumerate}
\item 按顺序检查每个客户是否可以获得服务
\item 能满足的客户完成交易
\item 不能满足的客户重新排队
\item 当银行资金不再增加或处理一定数量客户后停止检查
\end{enumerate}

\section{系统实现}
\subsection{错误处理机制}
系统采用统一的错误码机制,所有函数返回值都遵循以下规范:
\begin{itemize}
\item \texttt{BANK\_SUCCESS}: 操作成功
\item \texttt{BANK\_ERROR\_INVALID\_PARAM}: 无效参数
\item \texttt{BANK\_ERROR\_MEMORY\_ALLOC}: 内存分配失败
\item \texttt{BANK\_ERROR\_FILE\_OP}: 文件操作失败
\item \texttt{BANK\_ERROR\_QUEUE\_OP}: 队列操作失败
\item \texttt{BANK\_ERROR\_EVENT\_OP}: 事件操作失败
\end{itemize}

\subsection{日志系统}
系统实现了分级日志系统,支持DEBUG、INFO、WARN、ERROR四个级别,可通过配置文件或命令行参数控制日志级别和输出位置。

\subsection{配置管理}
系统支持通过配置文件进行参数配置,配置文件格式如下:

```
# 银行模拟系统配置文件

# 初始资金
initial_money = 10000

# 营业时间(分钟)
close_time = 600

# 是否启用日志
enable_log = 1

# 日志级别 (0=DEBUG, 1=INFO, 2=WARN, 3=ERROR)
log_level = 1

# 日志文件路径
log_file = bank_simulation.log
\end{verbatim}

\subsection{主要功能模块}
\subsubsection{队列操作}
\begin{itemize}
\item \texttt{init\_queue}: 初始化队列
\item \texttt{is\_queue\_empty}: 检查队列是否为空
\item \texttt{get\_queue\_size}: 获取队列大小
\item \texttt{enqueue}: 入队操作
\item \texttt{dequeue}: 出队操作
\item \texttt{queue\_front}: 获取队首元素
\end{itemize}

\subsubsection{事件管理}
\begin{itemize}
\item \texttt{insert\_event}: 插入事件(按时间顺序)
\item \texttt{get\_next\_event}: 获取下一个事件
\end{itemize}

\subsubsection{客户管理}
\begin{itemize}
\item \texttt{create\_customer}: 创建随机客户
\item \texttt{destroy\_customer}: 销毁客户
\item \texttt{handle\_arrive\_event}: 处理客户到达事件
\item \texttt{handle\_leave\_event}: 处理客户离开事件
\item \texttt{process\_queue2}: 处理等待队列
\end{itemize}

\subsubsection{系统控制}
\begin{itemize}
\item \texttt{init\_bank\_system}: 初始化银行系统
\item \texttt{destroy\_bank\_system}: 销毁银行系统
\item \texttt{run\_simulation}: 运行模拟
\item \texttt{print\_statistics}: 打印统计信息
\item \texttt{bank\_log}: 写日志
\item \texttt{load\_config}: 加载配置文件
\end{itemize}

\subsection{关键处理逻辑}
\subsubsection{客户到达处理}
\begin{enumerate}
\item 创建客户对象,随机生成服务时间和交易金额
\item 将客户加入第一个队列
\item 生成下一个客户到达事件
\end{enumerate}

\subsubsection{客户离开处理}
\begin{enumerate}
\item 计算客户逗留时间并累加
\item 根据业务类型处理:
  \begin{itemize}
  \item 存款:增加银行资金,处理等待队列
  \item 取款:检查资金是否充足,充足则完成,不足则加入等待队列
  \end{itemize}
\end{enumerate}

\subsubsection{等待队列处理}
\begin{enumerate}
\item 按顺序检查等待队列中的客户
\item 资金充足的客户完成交易
\item 资金仍不足的客户重新排队
\item 适当条件下停止处理以避免无限循环
\end{enumerate}

\section{测试与分析}
\subsection{测试方案}
\subsubsection{测试数据设置}
\begin{itemize}
\item 初始资金:10000元
\item 营业时间:600分钟
\item 客户到达间隔:1-10分钟随机
\item 服务时间:1-20分钟随机
\item 交易金额:-5000到5000元随机
\end{itemize}

\subsubsection{极端情况测试}
\begin{enumerate}
\item 客户到达间隔短,服务时间长
\item 客户到达间隔长,服务时间短
\end{enumerate}

\subsection{运行结果}
经过测试运行,系统能够正确处理所有客户事件,准确计算客户平均逗留时间,并合理处理资金不足情况。

\subsubsection{典型运行结果}
\begin{verbatim}
Bank Simulation System
======================
Initial money: 10000
Close time: 600 minutes
Logging enabled, level: 0, file: bank_simulation.log

Running simulation...
========================
Simulation completed, statistics:
Total customers: 114
Total wait time: 33797 minutes
Average stay time: 296.46 minutes
Final bank money: 10000 yuan
Remaining customers in waiting queue: 0
\end{verbatim}

\subsection{性能分析}
\subsubsection{时间复杂度}
\begin{itemize}
\item 事件插入操作:$O(n)$,其中n为当前事件列表长度
\item 事件获取操作:$O(1)$
\item 队列操作:$O(1)$
\item 总体时间复杂度:$O(n^2)$,其中n为总事件数
\end{itemize}

\subsubsection{空间复杂度}
\begin{itemize}
\item 客户对象:$O(n)$,其中n为客户数
\item 事件列表:$O(n)$,其中n为事件数
\item 队列存储:$O(n)$,其中n为队列中客户数
\item 总体空间复杂度:$O(n)$
\end{itemize}

\section{总结与体会}
\subsection{系统特点}
\subsubsection{优点}
\begin{enumerate}
\item \textbf{事件驱动}:采用事件驱动模拟,时间效率高
\item \textbf{动态内存管理}:使用动态内存分配,内存使用灵活
\item \textbf{模块化设计}:功能模块划分清晰,便于维护和扩展
\item \textbf{完整统计}:提供详细的统计信息,便于分析
\item \textbf{错误处理}:完善的错误处理机制,提高系统健壮性
\item \textbf{日志系统}:分级日志系统,便于调试和监控
\item \textbf{配置管理}:支持配置文件,提高系统灵活性
\end{enumerate}

\subsubsection{可能的改进}
\begin{enumerate}
\item \textbf{图形界面}:可以添加图形界面以更直观地显示模拟过程
\item \textbf{参数配置文件}:可以通过配置文件设置模拟参数
\item \textbf{日志记录}:添加详细的日志记录功能
\item \textbf{多窗口支持}:实现真正的多窗口服务
\item \textbf{性能优化}:使用优先队列优化事件处理时间复杂度
\end{enumerate}

\subsection{收获与体会}
通过本次课程设计,我深入理解了事件驱动模拟的思想,掌握了队列、链表等数据结构的实际应用。在实现过程中,我遇到了许多挑战,如事件顺序处理、等待队列管理等,通过不断调试和优化,最终完成了系统的设计与实现。

这次课程设计不仅提高了我的编程能力,更重要的是让我学会了如何将理论知识应用到实际问题中,对数据结构的理解也更加深刻。特别是在优化代码结构、完善错误处理机制和实现日志系统方面,我学到了很多大厂开发的实践经验。

\section{参考文献}
\begin{enumerate}
\item 吴伟民等. 《数据结构》. 广东工业大学计算机学院,2015.1
\item 严蔚敏, 吴伟民. 《数据结构(C语言版)》. 清华大学出版社
\item 数据结构实验指导书(2015春)
\end{enumerate}

\section{附录}
\subsection{源代码}
\subsubsection{bank\_simulation.h}
\begin{lstlisting}[language=C,caption=bank\_simulation.h]
/**
 * @file bank_simulation.h
 * @brief 银行模拟系统头文件
 * @author Yang Hengyi
 * @version 1.0
 * @date 2025-11-27
 */

#ifndef BANK_SIMULATION_H
#define BANK_SIMULATION_H

#include <stdio.h>
#include <stdlib.h>
#include <time.h>
#include <string.h>
#include <stdarg.h>

#ifdef __cplusplus
extern "C" {
#endif

// 常量定义
#define MAX_LINE_LENGTH 1024
#define DEFAULT_INITIAL_MONEY 10000
#define DEFAULT_CLOSE_TIME 600
#define MIN_SERVICE_TIME 1
#define MAX_SERVICE_TIME 20
#define MIN_ARRIVAL_INTERVAL 1
#define MAX_ARRIVAL_INTERVAL 10
#define MIN_TRANSACTION_AMOUNT -5000
#define MAX_TRANSACTION_AMOUNT 5000

// 错误码定义
#define BANK_SUCCESS 0
#define BANK_ERROR_INVALID_PARAM -1
#define BANK_ERROR_MEMORY_ALLOC -2
#define BANK_ERROR_FILE_OP -3
#define BANK_ERROR_QUEUE_OP -4
#define BANK_ERROR_EVENT_OP -5

// 事件类型枚举
typedef enum {
    EVENT_ARRIVE = 0,  // 客户到达事件
    EVENT_LEAVE = 1    // 客户离开事件
} EventType;

// 客户业务类型枚举
typedef enum {
    BUSINESS_WITHDRAW = 0,  // 取款/借款
    BUSINESS_DEPOSIT = 1    // 存款/还款
} BusinessType;

// 日志级别枚举
typedef enum {
    LOG_LEVEL_DEBUG = 0,
    LOG_LEVEL_INFO = 1,
    LOG_LEVEL_WARN = 2,
    LOG_LEVEL_ERROR = 3
} LogLevel;

// 客户结构体
typedef struct Customer {
    int id;                 // 客户ID
    int arrive_time;        // 到达时间
    int duration;           // 服务时间
    int amount;             // 交易金额(正数表示存款,负数表示取款)
    struct Customer* next;  // 链表指针
} Customer;

// 事件结构体
typedef struct Event {
    int occur_time;              // 事件发生时间
    EventType event_type;        // 事件类型
    struct Customer* customer;   // 相关客户
    struct Event* next;          // 链表指针
} Event;

// 队列结构体
typedef struct Queue {
    Customer* front;  // 队首指针
    Customer* rear;   // 队尾指针
    int size;         // 队列大小
} Queue;

// 银行系统配置结构体
typedef struct BankConfig {
    int initial_money;      // 初始资金
    int close_time;         // 关门时间
    int enable_log;         // 是否启用日志
    LogLevel log_level;     // 日志级别
    char log_file[256];     // 日志文件路径
} BankConfig;

// 银行系统结构体
typedef struct BankSystem {
    int total_money;        // 银行总资金
    int current_time;       // 当前时间
    int close_time;         // 关门时间
    int customer_count;     // 客户计数器
    Queue queue1;           // 第一个队列(正在等待服务的客户)
    Queue queue2;           // 第二个队列(因资金不足而等待的客户)
    Event* event_list;      // 事件列表
    int total_customers;    // 总客户数
    long long total_wait_time;  // 总等待时间
    FILE* log_fp;           // 日志文件指针
    LogLevel log_level;     // 日志级别
} BankSystem;

// 函数声明

/**
 * @brief 初始化银行系统
 * @param bank 银行系统指针
 * @param config 配置信息
 * @return 成功返回BANK_SUCCESS,失败返回错误码
 */
int init_bank_system(BankSystem* bank, const BankConfig* config);

/**
 * @brief 销毁银行系统
 * @param bank 银行系统指针
 * @return 成功返回BANK_SUCCESS,失败返回错误码
 */
int destroy_bank_system(BankSystem* bank);

/**
 * @brief 初始化队列
 * @param q 队列指针
 * @return 成功返回BANK_SUCCESS,失败返回错误码
 */
int init_queue(Queue* q);

/**
 * @brief 检查队列是否为空
 * @param q 队列指针
 * @return 空返回1,非空返回0
 */
int is_queue_empty(const Queue* q);

/**
 * @brief 获取队列大小
 * @param q 队列指针
 * @return 队列大小
 */
int get_queue_size(const Queue* q);

/**
 * @brief 入队
 * @param q 队列指针
 * @param customer 客户指针
 * @return 成功返回BANK_SUCCESS,失败返回错误码
 */
int enqueue(Queue* q, Customer* customer);

/**
 * @brief 出队
 * @param q 队列指针
 * @return 客户指针,失败返回NULL
 */
Customer* dequeue(Queue* q);

/**
 * @brief 获取队首元素
 * @param q 队列指针
 * @return 客户指针,失败返回NULL
 */
Customer* queue_front(const Queue* q);

/**
 * @brief 插入事件(按时间顺序)
 * @param event_list 事件列表指针的指针
 * @param event 事件指针
 * @return 成功返回BANK_SUCCESS,失败返回错误码
 */
int insert_event(Event** event_list, Event* event);

/**
 * @brief 获取下一个事件
 * @param event_list 事件列表指针的指针
 * @return 事件指针,失败返回NULL
 */
Event* get_next_event(Event** event_list);

/**
 * @brief 创建客户
 * @param id 客户ID
 * @param arrive_time 到达时间
 * @return 客户指针,失败返回NULL
 */
Customer* create_customer(int id, int arrive_time);

/**
 * @brief 销毁客户
 * @param customer 客户指针
 * @return 成功返回BANK_SUCCESS,失败返回错误码
 */
int destroy_customer(Customer* customer);

/**
 * @brief 处理客户到达事件
 * @param bank 银行系统指针
 * @param event 事件指针
 * @return 成功返回BANK_SUCCESS,失败返回错误码
 */
int handle_arrive_event(BankSystem* bank, Event* event);

/**
 * @brief 处理客户离开事件
 * @param bank 银行系统指针
 * @param event 事件指针
 * @return 成功返回BANK_SUCCESS,失败返回错误码
 */
int handle_leave_event(BankSystem* bank, Event* event);

/**
 * @brief 处理等待队列中的客户
 * @param bank 银行系统指针
 * @return 成功返回BANK_SUCCESS,失败返回错误码
 */
int process_queue2(BankSystem* bank);

/**
 * @brief 运行模拟
 * @param bank 银行系统指针
 * @return 成功返回BANK_SUCCESS,失败返回错误码
 */
int run_simulation(BankSystem* bank);

/**
 * @brief 打印统计信息
 * @param bank 银行系统指针
 * @return 成功返回BANK_SUCCESS,失败返回错误码
 */
int print_statistics(const BankSystem* bank);

/**
 * @brief 写日志
 * @param bank 银行系统指针
 * @param level 日志级别
 * @param format 格式化字符串
 * @param ... 可变参数
 * @return 成功返回BANK_SUCCESS,失败返回错误码
 */
int bank_log(const BankSystem* bank, LogLevel level, const char* format, ...);

/**
 * @brief 加载配置文件
 * @param config 配置结构体指针
 * @param filename 配置文件名
 * @return 成功返回BANK_SUCCESS,失败返回错误码
 */
int load_config(BankConfig* config, const char* filename);

#ifdef __cplusplus
}
#endif

#endif // BANK_SIMULATION_H
\end{lstlisting}

\subsubsection{bank\_simulation.c}
\begin{lstlisting}[language=C,caption=bank\_simulation.c]
/**
 * @file bank_simulation.c
 * @brief 银行模拟系统实现文件
 * @author Yang Hengyi
 * @version 1.0
 * @date 2025-11-27
 */

#include "../include/bank_simulation.h"
#include <stdarg.h>
#include <math.h>

// 日志级别字符串映射
static const char* log_level_strings[] = {
    "DEBUG",
    "INFO",
    "WARN",
    "ERROR"
};

/**
 * @brief 初始化银行系统
 */
int init_bank_system(BankSystem* bank, const BankConfig* config) {
    if (!bank || !config) {
        return BANK_ERROR_INVALID_PARAM;
    }

    // 初始化银行系统参数
    bank->total_money = config->initial_money;
    bank->current_time = 0;
    bank->close_time = config->close_time;
    bank->total_customers = 0;
    bank->total_wait_time = 0;
    bank->customer_count = 0;
    bank->event_list = NULL;

    // 初始化队列
    if (init_queue(&bank->queue1) != BANK_SUCCESS) {
        return BANK_ERROR_QUEUE_OP;
    }
    
    if (init_queue(&bank->queue2) != BANK_SUCCESS) {
        return BANK_ERROR_QUEUE_OP;
    }

    // 初始化日志
    bank->log_level = config->log_level;
    if (config->enable_log && strlen(config->log_file) > 0) {
        bank->log_fp = fopen(config->log_file, "w");
        if (!bank->log_fp) {
            fprintf(stderr, "Warning: Failed to open log file %s\n", config->log_file);
            bank->log_fp = stdout;
        }
    } else {
        bank->log_fp = NULL;
    }

    // 创建第一个客户到达事件
    Customer* first_customer = create_customer(bank->customer_count++, 0);
    if (!first_customer) {
        return BANK_ERROR_MEMORY_ALLOC;
    }

    Event* first_event = (Event*)malloc(sizeof(Event));
    if (!first_event) {
        destroy_customer(first_customer);
        return BANK_ERROR_MEMORY_ALLOC;
    }

    first_event->occur_time = 0;
    first_event->event_type = EVENT_ARRIVE;
    first_event->customer = first_customer;
    first_event->next = NULL;

    if (insert_event(&bank->event_list, first_event) != BANK_SUCCESS) {
        free(first_event);
        destroy_customer(first_customer);
        return BANK_ERROR_EVENT_OP;
    }

    bank_log(bank, LOG_LEVEL_INFO, "Bank system initialized with initial money: %d, close time: %d", 
             config->initial_money, config->close_time);

    return BANK_SUCCESS;
}

/**
 * @brief 销毁银行系统
 */
int destroy_bank_system(BankSystem* bank) {
    if (!bank) {
        return BANK_ERROR_INVALID_PARAM;
    }

    // 释放所有事件
    while (bank->event_list) {
        Event* event = get_next_event(&bank->event_list);
        if (event) {
            if (event->customer) {
                destroy_customer(event->customer);
            }
            free(event);
        }
    }

    // 释放队列中的所有客户
    Customer* customer;
    while ((customer = dequeue(&bank->queue1)) != NULL) {
        destroy_customer(customer);
    }

    while ((customer = dequeue(&bank->queue2)) != NULL) {
        destroy_customer(customer);
    }

    // 关闭日志文件
    if (bank->log_fp && bank->log_fp != stdout) {
        fclose(bank->log_fp);
    }

    bank_log(bank, LOG_LEVEL_INFO, "Bank system destroyed");
    return BANK_SUCCESS;
}

/**
 * @brief 初始化队列
 */
int init_queue(Queue* q) {
    if (!q) {
        return BANK_ERROR_INVALID_PARAM;
    }
    q->front = NULL;
    q->rear = NULL;
    q->size = 0;
    return BANK_SUCCESS;
}

/**
 * @brief 检查队列是否为空
 */
int is_queue_empty(const Queue* q) {
    if (!q) {
        return 1;
    }
    return q->front == NULL;
}

/**
 * @brief 获取队列大小
 */
int get_queue_size(const Queue* q) {
    if (!q) {
        return 0;
    }
    return q->size;
}

/**
 * @brief 入队
 */
int enqueue(Queue* q, Customer* customer) {
    if (!q || !customer) {
        return BANK_ERROR_INVALID_PARAM;
    }

    customer->next = NULL;

    if (is_queue_empty(q)) {
        q->front = customer;
        q->rear = customer;
    } else {
        q->rear->next = customer;
        q->rear = customer;
    }

    q->size++;
    return BANK_SUCCESS;
}

/**
 * @brief 出队
 */
Customer* dequeue(Queue* q) {
    if (!q || is_queue_empty(q)) {
        return NULL;
    }

    Customer* customer = q->front;
    q->front = q->front->next;

    if (q->front == NULL) {
        q->rear = NULL;
    }

    customer->next = NULL;
    q->size--;

    return customer;
}

/**
 * @brief 获取队首元素
 */
Customer* queue_front(const Queue* q) {
    if (!q || is_queue_empty(q)) {
        return NULL;
    }
    return q->front;
}

/**
 * @brief 插入事件(按时间顺序)
 */
int insert_event(Event** event_list, Event* event) {
    if (!event_list || !event) {
        return BANK_ERROR_INVALID_PARAM;
    }

    // 如果事件列表为空或新事件时间早于第一个事件
    if (*event_list == NULL || (*event_list)->occur_time > event->occur_time) {
        event->next = *event_list;
        *event_list = event;
        return BANK_SUCCESS;
    }

    // 查找插入位置
    Event* current = *event_list;
    while (current->next != NULL && current->next->occur_time <= event->occur_time) {
        current = current->next;
    }

    event->next = current->next;
    current->next = event;
    return BANK_SUCCESS;
}

/**
 * @brief 获取下一个事件
 */
Event* get_next_event(Event** event_list) {
    if (!event_list || *event_list == NULL) {
        return NULL;
    }

    Event* event = *event_list;
    *event_list = (*event_list)->next;
    event->next = NULL;
    return event;
}

/**
 * @brief 创建客户
 */
Customer* create_customer(int id, int arrive_time) {
    Customer* customer = (Customer*)malloc(sizeof(Customer));
    if (!customer) {
        return NULL;
    }

    customer->id = id;
    customer->arrive_time = arrive_time;

    // 随机生成服务时间(1-20分钟)
    customer->duration = rand() % (MAX_SERVICE_TIME - MIN_SERVICE_TIME + 1) + MIN_SERVICE_TIME;

    // 随机生成交易金额(-5000到5000元)
    customer->amount = rand() % (MAX_TRANSACTION_AMOUNT - MIN_TRANSACTION_AMOUNT + 1) + MIN_TRANSACTION_AMOUNT;

    customer->next = NULL;

    return customer;
}

/**
 * @brief 销毁客户
 */
int destroy_customer(Customer* customer) {
    if (!customer) {
        return BANK_ERROR_INVALID_PARAM;
    }
    free(customer);
    return BANK_SUCCESS;
}

/**
 * @brief 处理客户到达事件
 */
int handle_arrive_event(BankSystem* bank, Event* event) {
    if (!bank || !event || !event->customer) {
        return BANK_ERROR_INVALID_PARAM;
    }

    Customer* customer = event->customer;

    // 将客户加入第一个队列
    if (enqueue(&bank->queue1, customer) != BANK_SUCCESS) {
        bank_log(bank, LOG_LEVEL_ERROR, "Failed to enqueue customer %d", customer->id);
        destroy_customer(customer);
        free(event);
        return BANK_ERROR_QUEUE_OP;
    }

    bank->total_customers++;

    bank_log(bank, LOG_LEVEL_INFO, "Time %d: Customer %d arrived, transaction amount %d, service time %d", 
             bank->current_time, customer->id, customer->amount, customer->duration);

    // 生成下一个客户到达事件
    if (bank->current_time < bank->close_time) {
        // 下一个客户到达的时间间隔(1-10分钟)
        int interval = rand() % (MAX_ARRIVAL_INTERVAL - MIN_ARRIVAL_INTERVAL + 1) + MIN_ARRIVAL_INTERVAL;
        int next_arrive_time = bank->current_time + interval;

        if (next_arrive_time < bank->close_time) {
            Customer* next_customer = create_customer(bank->customer_count++, next_arrive_time);
            if (next_customer) {
                Event* next_event = (Event*)malloc(sizeof(Event));
                if (next_event) {
                    next_event->occur_time = next_arrive_time;
                    next_event->event_type = EVENT_ARRIVE;
                    next_event->customer = next_customer;
                    next_event->next = NULL;

                    if (insert_event(&bank->event_list, next_event) != BANK_SUCCESS) {
                        bank_log(bank, LOG_LEVEL_WARN, "Failed to insert next arrival event");
                        destroy_customer(next_customer);
                        free(next_event);
                        // 不返回错误,因为主流程仍然可以继续
                    }
                } else {
                    destroy_customer(next_customer);
                }
            }
        }
    }

    free(event);
    return BANK_SUCCESS;
}

/**
 * @brief 处理客户离开事件
 */
int handle_leave_event(BankSystem* bank, Event* event) {
    if (!bank || !event || !event->customer) {
        return BANK_ERROR_INVALID_PARAM;
    }

    Customer* customer = event->customer;

    // 计算客户在银行的总逗留时间
    int stay_time = bank->current_time - customer->arrive_time;
    bank->total_wait_time += stay_time;

    bank_log(bank, LOG_LEVEL_INFO, "Time %d: Customer %d leaving, transaction amount %d, total stay time %d", 
             bank->current_time, customer->id, customer->amount, stay_time);

    // 如果是存款或还款业务(金额为正)
    if (customer->amount > 0) {
        bank->total_money += customer->amount;
        bank_log(bank, LOG_LEVEL_INFO, "  Bank money increased by %d, current total %d", 
                 customer->amount, bank->total_money);

        // 处理第二个队列中等待的客户
        process_queue2(bank);
    } 
    // 如果是取款或借款业务(金额为负)
    else {
        // 检查银行是否有足够资金
        if (bank->total_money >= abs(customer->amount)) {
            bank->total_money += customer->amount; // amount为负数
            bank_log(bank, LOG_LEVEL_INFO, "  Bank money decreased by %d, current total %d", 
                     abs(customer->amount), bank->total_money);
        } else {
            // 资金不足,将客户加入第二个队列
            bank_log(bank, LOG_LEVEL_INFO, "  Insufficient funds, customer %d added to waiting queue", customer->id);
            if (enqueue(&bank->queue2, customer) != BANK_SUCCESS) {
                bank_log(bank, LOG_LEVEL_ERROR, "Failed to enqueue customer %d to waiting queue", customer->id);
                destroy_customer(customer);
                free(event);
                return BANK_ERROR_QUEUE_OP;
            }
            free(event);
            return BANK_SUCCESS;
        }
    }

    destroy_customer(customer);
    free(event);
    return BANK_SUCCESS;
}

/**
 * @brief 处理第二个队列中的客户
 */
int process_queue2(BankSystem* bank) {
    if (!bank) {
        return BANK_ERROR_INVALID_PARAM;
    }
    
    if (is_queue_empty(&bank->queue2)) {
        return BANK_SUCCESS;
    }

    bank_log(bank, LOG_LEVEL_DEBUG, "Processing waiting queue customers");

    // 记录处理前的银行资金
    int money_before = bank->total_money;
    int processed_count = 0;

    // 创建临时队列存储无法处理的客户
    Queue temp_queue;
    if (init_queue(&temp_queue) != BANK_SUCCESS) {
        bank_log(bank, LOG_LEVEL_ERROR, "Failed to initialize temp queue");
        return BANK_ERROR_QUEUE_OP;
    }

    Customer* customer = dequeue(&bank->queue2);
    while (customer != NULL) {
        // 检查银行是否有足够资金处理该客户
        if (bank->total_money >= abs(customer->amount)) {
            // 可以处理
            bank->total_money += customer->amount;
            int stay_time = bank->current_time - customer->arrive_time;
            bank->total_wait_time += stay_time;
            processed_count++;

            bank_log(bank, LOG_LEVEL_INFO, "  Time %d: Waiting customer %d completed transaction, amount %d, total stay time %d", 
                     bank->current_time, customer->id, customer->amount, stay_time);
            bank_log(bank, LOG_LEVEL_INFO, "  Bank money changed by %d, current total %d", 
                     customer->amount, bank->total_money);

            destroy_customer(customer);
        } else {
            // 仍然无法处理,放回临时队列
            if (enqueue(&temp_queue, customer) != BANK_SUCCESS) {
                bank_log(bank, LOG_LEVEL_ERROR, "Failed to enqueue customer %d to temp queue", customer->id);
                destroy_customer(customer);
            }
        }

        // 检查是否应该停止处理
        if (bank->total_money <= money_before || processed_count > 100) {
            // 将当前客户重新放回队列
            if (customer != NULL && enqueue(&temp_queue, customer) != BANK_SUCCESS) {
                bank_log(bank, LOG_LEVEL_ERROR, "Failed to enqueue customer %d to temp queue", customer->id);
                destroy_customer(customer);
            }
            break;
        }

        customer = dequeue(&bank->queue2);
    }

    // 将未处理的客户移回等待队列
    while (!is_queue_empty(&temp_queue)) {
        Customer* temp_customer = dequeue(&temp_queue);
        if (temp_customer && enqueue(&bank->queue2, temp_customer) != BANK_SUCCESS) {
            bank_log(bank, LOG_LEVEL_ERROR, "Failed to enqueue customer %d back to waiting queue", temp_customer->id);
            destroy_customer(temp_customer);
        }
    }

    bank_log(bank, LOG_LEVEL_DEBUG, "Finished processing waiting queue");
    return BANK_SUCCESS;
}

/**
 * @brief 运行模拟
 */
int run_simulation(BankSystem* bank) {
    if (!bank) {
        return BANK_ERROR_INVALID_PARAM;
    }

    bank_log(bank, LOG_LEVEL_INFO, "Bank simulation started");
    bank_log(bank, LOG_LEVEL_INFO, "Initial money: %d", bank->total_money);
    bank_log(bank, LOG_LEVEL_INFO, "Close time: %d", bank->close_time);
    bank_log(bank, LOG_LEVEL_INFO, "========================");

    while (bank->event_list != NULL) {
        Event* event = get_next_event(&bank->event_list);
        if (event == NULL) {
            break;
        }

        bank->current_time = event->occur_time;

        // 如果已经超过关门时间,结束营业
        if (bank->current_time >= bank->close_time) {
            bank_log(bank, LOG_LEVEL_INFO, "Time %d: Bank closed, business ended", bank->current_time);
            if (event->customer) {
                destroy_customer(event->customer);
            }
            free(event);
            break;
        }

        int result;
        if (event->event_type == EVENT_ARRIVE) {
            // 到达事件
            result = handle_arrive_event(bank, event);
        } else {
            // 离开事件
            result = handle_leave_event(bank, event);
        }
        
        // 如果处理事件时出现严重错误,停止模拟
        if (result != BANK_SUCCESS && result != BANK_ERROR_EVENT_OP) {
            bank_log(bank, LOG_LEVEL_ERROR, "Error occurred during event processing, stopping simulation");
            return result;
        }
    }

    // 处理完所有事件后,处理队列1中剩余的客户
    bank_log(bank, LOG_LEVEL_INFO, "========================");
    bank_log(bank, LOG_LEVEL_INFO, "Processing remaining customers");

    while (!is_queue_empty(&bank->queue1)) {
        Customer* customer = dequeue(&bank->queue1);
        if (customer) {
            int stay_time = bank->current_time - customer->arrive_time;
            bank->total_wait_time += stay_time;
            bank_log(bank, LOG_LEVEL_INFO, "Time %d: Customer %d left (business ended), total stay time %d", 
                     bank->current_time, customer->id, stay_time);
            destroy_customer(customer);
        }
    }

    return BANK_SUCCESS;
}

/**
 * @brief 打印统计信息
 */
int print_statistics(const BankSystem* bank) {
    if (!bank) {
        return BANK_ERROR_INVALID_PARAM;
    }

    printf("========================\n");
    printf("Simulation completed, statistics:\n");
    printf("Total customers: %d\n", bank->total_customers);
    printf("Total wait time: %lld minutes\n", bank->total_wait_time);

    if (bank->total_customers > 0) {
        double avg_stay_time = (double)bank->total_wait_time / bank->total_customers;
        printf("Average stay time: %.2f minutes\n", avg_stay_time);
    }

    printf("Final bank money: %d yuan\n", bank->total_money);
    printf("Remaining customers in waiting queue: %d\n", get_queue_size(&bank->queue2));
    return BANK_SUCCESS;
}

/**
 * @brief 写日志
 */
int bank_log(const BankSystem* bank, LogLevel level, const char* format, ...) {
    if (!bank || level < bank->log_level) {
        return BANK_SUCCESS;
    }

    // 如果没有启用日志,直接返回
    if (!bank->log_fp) {
        return BANK_SUCCESS;
    }

    // 获取当前时间
    time_t now;
    time(&now);
    struct tm* local_time = localtime(&now);

    // 写入日志级别和时间
    fprintf(bank->log_fp, "[%04d-%02d-%02d %02d:%02d:%02d][%s] ", 
            local_time->tm_year + 1900,
            local_time->tm_mon + 1,
            local_time->tm_mday,
            local_time->tm_hour,
            local_time->tm_min,
            local_time->tm_sec,
            log_level_strings[level]);

    // 写入日志内容
    va_list args;
    va_start(args, format);
    vfprintf(bank->log_fp, format, args);
    va_end(args);

    fprintf(bank->log_fp, "\n");
    fflush(bank->log_fp);
    return BANK_SUCCESS;
}

/**
 * @brief 加载配置文件
 */
int load_config(BankConfig* config, const char* filename) {
    if (!config || !filename) {
        return BANK_ERROR_INVALID_PARAM;
    }

    FILE* fp = fopen(filename, "r");
    if (!fp) {
        return BANK_ERROR_FILE_OP;
    }

    char line[MAX_LINE_LENGTH];
    while (fgets(line, sizeof(line), fp)) {
        // 跳过注释和空行
        if (line[0] == '#' || line[0] == '\n' || line[0] == '\r') {
            continue;
        }

        char key[MAX_LINE_LENGTH];
        char value[MAX_LINE_LENGTH];

        if (sscanf(line, "%s = %s", key, value) == 2) {
            if (strcmp(key, "initial_money") == 0) {
                config->initial_money = atoi(value);
            } else if (strcmp(key, "close_time") == 0) {
                config->close_time = atoi(value);
            } else if (strcmp(key, "enable_log") == 0) {
                config->enable_log = atoi(value);
            } else if (strcmp(key, "log_level") == 0) {
                config->log_level = (LogLevel)atoi(value);
            } else if (strcmp(key, "log_file") == 0) {
                strncpy(config->log_file, value, sizeof(config->log_file) - 1);
                config->log_file[sizeof(config->log_file) - 1] = '\0';
            }
        }
    }

    fclose(fp);
    return BANK_SUCCESS;
}
\end{lstlisting}

\subsubsection{main.c}
\begin{lstlisting}[language=C,caption=main.c]
/**
 * @file main.c
 * @brief 银行模拟系统主程序
 * @author Yang Hengyi
 * @version 1.0
 * @date 2025-11-27
 */

#include "../include/bank_simulation.h"
#include <unistd.h>

/**
 * @brief 打印帮助信息
 */
void print_help(const char* program_name) {
    printf("Usage: %s [options]\n", program_name);
    printf("Options:\n");
    printf("  -h, --help              Show this help message\n");
    printf("  -c, --config <file>     Specify configuration file\n");
    printf("  -m, --money <amount>    Set initial money (default: 10000)\n");
    printf("  -t, --time <minutes>    Set close time in minutes (default: 600)\n");
    printf("  -l, --log               Enable logging\n");
    printf("  -v, --verbose           Enable verbose logging (DEBUG level)\n");
    printf("\n");
    printf("Examples:\n");
    printf("  %s                          # Run with default settings\n", program_name);
    printf("  %s -m 5000 -t 300          # Run with 5000 initial money and 300 minutes\n", program_name);
    printf("  %s -c config.ini           # Run with configuration file\n", program_name);
    printf("  %s -l -v                   # Run with verbose logging\n", program_name);
}

/**
 * @brief 创建默认配置文件
 */
int create_default_config(const char* filename) {
    FILE* fp = fopen(filename, "w");
    if (!fp) {
        return BANK_ERROR_FILE_OP;
    }

    fprintf(fp, "# 银行模拟系统配置文件\n");
    fprintf(fp, "\n");
    fprintf(fp, "# 初始资金\n");
    fprintf(fp, "initial_money = 10000\n");
    fprintf(fp, "\n");
    fprintf(fp, "# 营业时间(分钟)\n");
    fprintf(fp, "close_time = 600\n");
    fprintf(fp, "\n");
    fprintf(fp, "# 是否启用日志\n");
    fprintf(fp, "enable_log = 1\n");
    fprintf(fp, "\n");
    fprintf(fp, "# 日志级别 (0=DEBUG, 1=INFO, 2=WARN, 3=ERROR)\n");
    fprintf(fp, "log_level = 1\n");
    fprintf(fp, "\n");
    fprintf(fp, "# 日志文件路径\n");
    fprintf(fp, "log_file = bank_simulation.log\n");

    fclose(fp);
    return BANK_SUCCESS;
}

/**
 * @brief 主函数
 */
int main(int argc, char* argv[]) {
    // 设置随机数种子
    srand((unsigned int)time(NULL));

    // 初始化默认配置
    BankConfig config = {
        .initial_money = DEFAULT_INITIAL_MONEY,
        .close_time = DEFAULT_CLOSE_TIME,
        .enable_log = 0,
        .log_level = LOG_LEVEL_INFO,
        .log_file = ""
    };

    // 解析命令行参数
    int opt;
    while ((opt = getopt(argc, argv, "hc:m:t:lv")) != -1) {
        switch (opt) {
            case 'h':
                print_help(argv[0]);
                return 0;
            case 'c':
                if (load_config(&config, optarg) != BANK_SUCCESS) {
                    fprintf(stderr, "Error: Failed to load config file %s\n", optarg);
                    return 1;
                }
                break;
            case 'm':
                config.initial_money = atoi(optarg);
                break;
            case 't':
                config.close_time = atoi(optarg);
                break;
            case 'l':
                config.enable_log = 1;
                if (strlen(config.log_file) == 0) {
                    strncpy(config.log_file, "bank_simulation.log", sizeof(config.log_file) - 1);
                }
                break;
            case 'v':
                config.log_level = LOG_LEVEL_DEBUG;
                break;
            default:
                print_help(argv[0]);
                return 1;
        }
    }

    // 如果启用了日志但没有指定日志文件,使用默认文件名
    if (config.enable_log && strlen(config.log_file) == 0) {
        strncpy(config.log_file, "bank_simulation.log", sizeof(config.log_file) - 1);
    }

    // 创建默认配置文件(如果不存在)
    if (access("config.ini", F_OK) == -1) {
        if (create_default_config("config.ini") != BANK_SUCCESS) {
            fprintf(stderr, "Warning: Failed to create default config file\n");
        }
    }

    printf("Bank Simulation System\n");
    printf("======================\n");
    printf("Initial money: %d\n", config.initial_money);
    printf("Close time: %d minutes\n", config.close_time);
    if (config.enable_log) {
        printf("Logging enabled, level: %d, file: %s\n", config.log_level, config.log_file);
    }
    printf("\n");

    // 初始化银行系统
    BankSystem bank;
    if (init_bank_system(&bank, &config) != BANK_SUCCESS) {
        fprintf(stderr, "Error: Failed to initialize bank system\n");
        return 1;
    }

    // 运行模拟
    printf("Running simulation...\n");
    if (run_simulation(&bank) != BANK_SUCCESS) {
        fprintf(stderr, "Error: Failed to run simulation\n");
        destroy_bank_system(&bank);
        return 1;
    }

    // 打印统计信息
    if (print_statistics(&bank) != BANK_SUCCESS) {
        fprintf(stderr, "Error: Failed to print statistics\n");
        destroy_bank_system(&bank);
        return 1;
    }

    // 销毁银行系统
    if (destroy_bank_system(&bank) != BANK_SUCCESS) {
        fprintf(stderr, "Error: Failed to destroy bank system\n");
        return 1;
    }

    return 0;
}
\end{lstlisting}

\subsection{Makefile}
\begin{lstlisting}[caption=Makefile]
# 银行模拟系统 Makefile

# 编译器
CC = gcc

# 编译选项
CFLAGS = -Wall -Wextra -std=c99 -g -O2 -pedantic
CFLAGS += -D_DEFAULT_SOURCE  # 用于兼容较新的Linux系统

# 包含目录
INCLUDES = -Iinclude

# 目标文件
TARGET = bin/bank_simulation

# 源文件目录
SRC_DIR = src
INC_DIR = include
BIN_DIR = bin

# 源文件
SRCS = $(SRC_DIR)/bank_simulation.c $(SRC_DIR)/main.c

# 对象文件
OBJS = $(SRCS:$(SRC_DIR)/%.c=$(BIN_DIR)/%.o)

# 依赖文件
DEPS = $(OBJS:.o=.d)

# 默认目标
all: directories $(TARGET)

# 创建所需目录
directories:
	@mkdir -p $(BIN_DIR)

# 链接目标文件
$(TARGET): $(OBJS)
	$(CC) $(OBJS) -o $@
	@echo "Build successful: $@"

# 编译源文件(包含依赖关系)
$(BIN_DIR)/%.o: $(SRC_DIR)/%.c
	$(CC) $(CFLAGS) $(INCLUDES) -MMD -MP -c $< -o $@

# 包含自动生成的依赖文件
-include $(DEPS)

# 清理生成文件
clean:
	rm -rf $(BIN_DIR) $(TARGET)

# 重新编译
rebuild: clean all

# 运行程序
run: $(TARGET)
	./$(TARGET)

# 安装(示例)
install: $(TARGET)
	cp $(TARGET) /usr/local/bin/

# 显示变量(调试用)
print-vars:
	@echo "SRCS: $(SRCS)"
	@echo "OBJS: $(OBJS)"
	@echo "DEPS: $(DEPS)"

.PHONY: all clean rebuild run install print-vars directories
\end{lstlisting}

\end{document}
```

```
